\section{2018.9.10,12 Monday,Wednesday}
\begin{defi}
	Suppose $D\subset \m{C}$ is a domain, a function $f:D\to\m{C}$ is called analytic if
	\[f'(z):=\lim_{h\to 0} \frac{f(z+h)-f(z)}{h}\;\mbox{exists},\quad \forall z\in D.\]
	Let $\ma{O}(D):=\{f:D\to C: f\,\mbox{analytic}\}.$
\end{defi}
We would give a brief review of the under graduated complex analysis. From the calculus point of view, 
the Cauchy-Riemann equations link the analytic function and real function in two variables together.
From the integral point of view, Cauchy theorem and its inverse theorem play a fundamental role. 
\begin{ex}
	Use Cauchy theorem to show Cauchy integral formula and Cauchy inequation, then prove Liouville theorem
	and the algebra fundamental theorem. Notice that any internal closed uniform limit of a sequence of analytic
	functions is still analytic.
\end{ex}

From the algebraic and geometric point of view, Weierstrass pointed out that a function is analytic if and only if it can be locally expanded
to a power series. Two of the most important applications are Riemann removable singularity theorem and identity theorem, 
which is to say that any none constant analytic function has only isolated zeros. These two theorems can lead to the 
geometric characterization of analytic functions:
\begin{thm}
	Suppose $f:D\to \m{C}$ is $C^1$ then $f$ is analytic if and only if
	\begin{description}
		\item{(i)} the set of zeros $A$ of $\det(df)$ is discrete, and
		\item{(ii)} $f|_{D\ba A}: D\ba A\to \m{C}$ is isogonal.
	\end{description}
\end{thm}
Here isogonal means
\begin{defi}
	Let $T:\m{R}^2\to \m{R}^2$ be a linear isomorphism then
	\begin{description}
		\item{(i)} $T$ is isogonal if $\angle (T\al,T\beta)=\angle (\al,\beta),\;\forall \al,\beta$,
		\item{(ii)} $T$ is conformal if $\|T\al\|=\la \|\al\|$ for some $\la>0$ constant, $\forall \al$,
		\item{(iii)} $T$ is orientation preserved if $\det T>0$.
	\end{description}
	For $f\in\ma{O}(D)$, we say $f$ is isogonal/conformal/orientation preserved at $z$ if $df$ is
	isogonal/conformal/orientation preserved at $z$ respectively.
\end{defi}

From the topological point of view, we learned the open mapping theory. To show it, the following lemma is needed.
\begin{lem}
	Let $f$ analytic at $z_0$ with $f(z_0)\neq 0$, then for every $m\geq 1$ there exists some $g$ such that
	$f(z)=g^m(z)$ around $z_0$ and $g$ analytic at $z_0$.
\end{lem}
The Maximum Theory and Schwartz Lemma follow immediately and by which one can show
\begin{align*}
	\Aut(\D)&= \{f:\D \to \D: f\in \ma{O}(\D),\; f\;\mbox{bijective}\}\\
	       &= \left\{z\mapsto e^{i\cta} \frac{z-z_0}{1-\conj{z_0}z}: \cta\in\m{R},\, z_0\in\D\right\}\\
	       &\cong SU(1,1)\cong SL(2,\m{R})/{\pm 1},
\end{align*}
where $\D$ is the open unit disk.